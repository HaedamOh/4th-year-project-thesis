% My favourite part of theses are the acknowledgments sections. Yes, we spent at least 4 years working on research, published our papers, and attended conferences that are worth reporting on their own. But that is not everything that happened in those years. Acknowledgment sections give you a chance to learn that.

% \medskip

% First I thank my supervisor Maurice Fallon for replying to my emails while I was a master's student in Chile, and for giving me the chance to join the Dynamic Robot Systems (DRS) group. I thank the support and advice these 4 years, as well as showing me the importance of working with real robots.

% \medskip

% I am grateful to my examiners, Stefan Leutenegger and Ioannis Havoutis, for taking the time to read my work, as well as the interesting discussions and feedback they gave me during the viva. Stefan's expertise in state estimation as well as Ioannis' in legged robots made them a good match to assess the work presented in this thesis. I also thank Jonathan Gammel and Ingmar Posner for assessing my work in the Transfer and Confirmation of Status, and giving me valuable comments to move forward with my research.

% \medskip

% I would like to acknowledge all the friends and colleagues in DRS I met over the years. Milad Ramezani for encouraging me to solve the problems now and not later (and also helping me to move home!). Marco Camurri for his technical expertise, but also chats on food and \emph{nerdy} things (like The Cave). Russell Buchanan for sharing his experiences and knowledge that helped me so many times. David Rytz for the many technical (and not-so) conversations over a coffee, as well as painful debugging sessions overnight. David Wisth for developing VILENS and all I learned by just checking his code. Georgi Tinchev for our brief chats, but most importantly for being the only person of DRS I met in person before I applied. Yiduo Wang for the research discussions but also for sharing our passion for LEGOs. Lintong Zhang for the hard work, the fun, and Korean fried chicken. Ethan Tao for his ML expertise, book recommendations, and long conversations after work. Nived Chebrolu for his support, the laughs, and the pragmatic advice always needed. Tejaswi Digumarti for the wisdom, and nice recommendations when I visited Zurich. Frank Fu for the chance to discuss crazy ideas at a whiteboard, the adjoints, and \emph{pollito frito}. Christina Kassab for the collaborations, the socials, and bringing \emph{weird} ideas that were worth exploring. Joseph Rowell for the lunches and jokes, but also valuable feedback. Jianeng Wang for the collaborations and example of commitment. Jonas Beuchart for his advice and feedback on my thesis, but also for sharing the amazing whereabouts of his turtles. Rowan Border for the technical discussions and advice in the last bits of my PhD.

% \medskip

% When I started DRS was a huge group, but that changed over the years. Still, there are many people I am thankful to have shared time with. Mathieu Geisert gave me the chance to join my first field trial, but also we shared many chats after work in pubs with Siddhant Gangapurwala, Romeo Orsolino, David Surovik, and Luigi Campanaro, from which I learned many things about robot control, research, and life decisions. Wolfgang Merkt's technical expertise was always valuable when debugging problems or embarking on new software developments. I enjoyed working with Oliwier Melon and Alexander Mitchell (as well as David Rytz) to develop a new CDT challenge in 2021, from scratch, during COVID times, which we successfully executed (I hope).

% \medskip

% While I started the PhD in DRS with Russell and David, we also started in the Oxford Robotics Institute (ORI) with more people I am really grateful to met. I thank Mohamed Baioumy for always being keen to chat, discuss new ideas, and give me a chance to contribute since the beginning, despite the broken English I spoke when I started. I appreciated sharing an office and lunches with David Williams, who always had interesting questions to discuss and showed me how to make a \emph{proper} cup of English tea. With Paarth Shah I had so many chats about research and optimisation, as well as shared so many pictures of cats. And of course thanks to Wenye Ouyang, for being my flatmate for almost 2 years and sharing the pain of the lockdown but the joy of the \emph{Eat Out to Help Out}.

% \medskip

% The ORI engineering team contributed enormously to my experience and work these years and I cannot thank them enough. I thank Benoit Casseau for all the work and help in different projects, so many field trials, and all the times he allowed me to bother him with a new question. Michal Staniaszek always helped me to solve technical issues and I am happy to see he is a PhD student now. Chris Prahacs always provided wisdom and the finest support, especially during the hardest times of the pandemic. Tom Dobra always solved all my infrastructure questions. I thank Matt Towlson and Wayne Tubby for always supporting any new hardware endeavour, and for being happy to help me with any new enquiry I brought despite how annoying I was. Jon Ody helped me fix ANYmal B during my first hardware experiments, which was a crucial step I needed to finish my first paper in 2020 in spite of all adversity. I lastly thank Tobit Flatscher not only for the C++ and Docker expertise but also for his remarkable willingness to share his knowledge (and hand-made pizzas!).

% \medskip

% The ORI admin team made my life a lot easier these years. Rosemary Cameron introduced me to life here and solved all my initial questions. Laura O'Mahony provided all the support (many times under the hood) to make sure the processes worked. Oliver Barlett helped organise many of the first field trials I did in the ORI. Marysa Chapman helped me with the registration for the first conference I attended. Acacia Nockolds provided crucial support for my long stay in Zurich. Amber Allen helped me organise important field trials I needed for the RSS paper. Elsa Lam always helped me with \emph{any} admin questions I had, from trips to getting new hardware. Timea Thorpe and Daniel Marques made sure that everything worked flawlessly at any event, field trial, or public demo I participated in.

% \medskip 

% I thank all my other ORI colleagues, for the spontaneous chats in the kitchen, for free food, or for sharing the struggle of working late for a deadline.

% \medskip

% In 2022 I had the unique chance to visit the Robotic Systems Lab at ETH Zurich. I recognise it as one of the most important labs in robotics research nowadays and it was an honour to be there. I thank Marco Hutter for kindly allowing me to join them for 6 months and have a chance to contribute to his lab. To Maria Trodella for making this happen, helping me with all the procedures, and solving all my questions. And to Cesar Cadena for the support and sincere advice, which I particularly appreciate coming from a fellow South American.

% \medskip

% I met many researchers at RSL I learned a lot from them. But first of all, I would like to thank Jonas Frey, for being the best research collaborator I could have had. I really enjoyed working with him, and learned so many things just by jointly developing the Wild Visual Navigation project---improving my Python skills being one of them. I wish we can keep collaborating in the future. I also thank the other members of Room H317, the \emph{Navigation office}: Turcan Tuna, Fan Yang, Gabriel Waibel, Julian Nubert, and Viktor Klemm. I appreciate our chats, coffees, postcards, and also the weekend meetups at the Limmat. I also thank other people I had a chance to share and collaborate with: Simone Arreghini, Takahiro Miki, Joonho Lee, Yuntao Ma, Mayank Mittal, Philip Arm, Ruyi Zhou, Alessandro Fulciniti, Konrad Meyer, Markus Montenegro, Lorenz Wellhausen, Timon Homberger, and Marco Tranzatto.

% \medskip

% I also must thank the robots at the ORI and RSL: Oxford's ANYmal B ``Boxy'' (RIP), ANYmal C ``Coyote'', and RSL's ``Camel'' and ``Cerberus''. While you made me suffer, seeing you run the algorithms I implemented and navigate autonomously is one of my biggest satisfactions. This thesis would not be possible without you.

% \medskip

% These four years were not only about robots. I also met amazing non-robotics people. I thank my friends at Hertford College, Amelia Lee, Arjaan Buijs, and William Ivison for sharing our flat the first year before COVID hit, as well as Guopeng Chen, Thibault Jouen, Haylee He, Bruce Li, and Vít Růžička for the welfare walks, board game nights and Oxmas dinners. I also thank Amelia for being my tour guide in Seoul when I attended RSS.

% \medskip

% The Chilean community in Oxford is also larger than I expected, and I thank all the people I met over these years. Particularly, I thank Arantxa Gutiérrez, Thomas Püschel, and Gonzalo Mena for all the gatherings, laughs, advice, and support. I also appreciate the support of Fernanda Duarte and Raúl Santos, who were my neighbors at the beginning and we did not notice until weeks after I arrived.

% \medskip

% I also thank my other Chilean friends in Europe and around the globe. I did not have the chance to visit or meet all of them, but having the opportunity to share a few words or asking how we were doing, was highly valuable to me, particularly during COVID. I specially thank Daniela Barrientos, Patricio Correa, and Matias Suazo for the weekly meetings, online movie and game nights during the lockdown to stay sane, as well as the nice holidays we spent together once things got better.

% \medskip

% I thank my friends in Chile for supporting me in spite of the distance, and for making the time to meet with me when I (finally) could travel there to visit in 2021.

% \medskip

% Gracias a mi familia por el apoyo que me han dado desde siempre, aún sabiendo que no era fácil para ustedes ni para mi decidir estudiar en el extranjero por tanto tiempo. Gracias a mi mamá Mabel, mi papá Joel, mi hermano Tomás y mi abuelita Oli por el apoyo y cariño estos años, por las llamadas que teníamos cada semana, y por siempre preguntarme cómo estoy. Estoy esperando el día en que puedan viajar para mi graduación, y mostrarles los lugares y experiencias que viví estos 4 años. Los amo.

% \medskip

% Finally, I thank to my partner Pía Cortés for sharing with me this adventure from the very beginning. From our preparations for the English tests, searching for places where we could apply, going through the applications, rejections, or no responses, to the time it got real and we had to move ahead. While we did not manage to make things happen the way we wanted and ended up in completely different places for our PhDs, we made it work. Despite not having the chance to see us as often as we wanted, COVID, and other constraints, we made it work. And despite all the highs and lows of our own PhDs, we made it work. Let us keep making it work.\vspace*{\fill}


% This PhD was funded by the Chilean National Agency for Research and Development (ANID) / DOCTORADO BECAS CHILE/2019 - 72200291. The research visit to ETH Zurich was partially funded by NCCR Robotics. Other trips were partially funded by Hertford College and the Department of Engineering Science.
