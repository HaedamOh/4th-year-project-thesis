% % max 300 words

Many LiDAR place recognition systems have been developed and tested specifically for urban driving scenarios. Their performance in natural environments such as forests and woodlands have been studied less closely.
In this paper, we analyzed the capabilities of four different LiDAR place recognition systems, both handcrafted and learning-based methods, using LiDAR data collected with a handheld device and legged robot within dense forest environments. 
In particular, we focused on evaluating localization where there is significant translational and orientation difference between corresponding LiDAR scan pairs. This is particularly important for forest survey systems where the sensor or robot does not follow a defined road or path. 
Extending our analysis we then incorporated the best performing approach, Logg3dNet, into a full 6-DoF pose estimation system---introducing several verification layers for precise registration. 
We demonstrated the performance of our methods in three operational modes: online SLAM, offline multi-mission SLAM map merging, and relocalization into a prior map. 
We evaluated these modes using data captured in forests from three different countries, achieving \SI{80}{\percent} of correct loop closures candidates with baseline distances up to \SI{5}{\meter}, and \SI{60}{\percent} up to \SI{10}{\meter}.  \\

This work has been submitted to IEEE/RSJ International Conference on Intelligent Robots and Systems (IROS) 2024. Upon acceptance, the paper will be published in the conference proceedings. For future updates, you can access the paper through the following link\footnote{\url{https://arxiv.org/abs/2403.14326}} and video demonstration at \url{Here}.