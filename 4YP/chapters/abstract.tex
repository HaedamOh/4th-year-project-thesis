% % max 300 words

Many LiDAR place recognition systems have been
developed and tested specifically for urban driving scenarios.
Their capabilities in natural environments such as forests and
woodlands have been studied less closely. In this thesis, we
analyze the capabilities of four different LiDAR place recog-
nition systems, both handcrafted and learning-based methods,
using LiDAR data collected with a handheld device and legged
robot in dense forest environments. In particular, we focused on
evaluating localization where there is significant transnational
and orientation difference between corresponding LiDAR scan
pairs. This is particularly important for forest survey systems
where the sensor or robot does not follow a defined road or
path. 
Extending our analysis we then incorporated the best
performing approach (called Logg3dNet) into a full 6-DoF pose
estimation system (with several layers of outlier checks and
ICP for precise registration). We demonstrate performance in
three operational modes: online SLAM, offline multi-mission
SLAM map merging, and relocalization into a prior map.
Our place recognition performance analysis was carried out
using data captured in forest in three different countries. We demonstrate reliable loop closures upto 15\,m distance away inside dense forests.  \\

This work has been submitted to IEEE/RSJ International Conference on Intelligent Robots and Systems (IROS) 2024. Upon acceptance, the paper will be published in the conference proceedings. For future updates, you can access the paper through the following link: \url{Upload sometime} and video demonstration at \url{Upload sometime}.