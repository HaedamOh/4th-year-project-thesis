% % max 300 words
Many LiDAR place recognition systems have been developed and typically focused on self driving scenarios in urban environments. Their performance in natural environments such as forests and woodlands have been studied less closely. In this thesis, we analyze the capabilities of four different LiDAR place recognition systems, both handcrafted and learning-based methods, using LiDAR data collected with a handheld device and a legged robot within dense forest environments. 
In particular, we focused on evaluating localization performance where there is significant translational and orientation difference between corresponding LiDAR scan pairs. This property is particularly important for forest survey systems where the sensor or robot does not follow a defined road or path and revisiting previously visited paths would be impractical.  
Extending our analysis we then incorporated the best performing approach, Logg3dNet, into a full 6-DoF pose estimation system---introducing several verification layers to achieve precise and reliable registration.  
We demonstrated the performance of our methods in three operational modes: online SLAM, offline multi-mission SLAM map merging, and relocalization into a prior map. 
Then we evaluated these modes using data captured in forests from three different countries. The overall place recognition system could acheive \SI{80}{\percent} of correct loop closures candidates with baseline distances up to \SI{5}{\meter}, and \SI{60}{\percent} up to \SI{10}{\meter}.  \\

The work has been submitted to IEEE/RSJ International Conference on Intelligent Robots and Systems (IROS) 2024. A pre-print of the paper\footnote{\url{https://arxiv.org/abs/2403.14326}} and video\footnote{ \url{https://drive.google.com/file/d/1EjFU06T_RmUtu6fig0JufpV0g8TkXDfD/view?usp=drive_link}} are available online. 