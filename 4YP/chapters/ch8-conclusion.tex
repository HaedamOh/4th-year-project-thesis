\chapter{Conclusion}
\label{chap:conclusion}

In this thesis, we conducted extensive testing of LiDAR place recognition systems in dense forest environments. We presented a place recognition and verification system that leverages three stages of loop-candidate verification: at the global descriptor-level, local feature-level, and fine point cloud level. These place recognition modules were seamlessly integrated into a pose graph SLAM system and evaluated across three distinct scenarios: online SLAM, offline multi-mission SLAM, and relocalization. Our experiments provide further insights on the performance of currently available LiDAR-based place recognition methods in dense forests. Further, they demonstrate different integration cases to achieve 6-DoF localization, opening future applications for forest inventory, inspection, and autonomous tasks. 

In future work, we plan to refactor the system to be a separate server, allowing for easier integration of new LiDAR place recognition methods and available for DRS members. We also plan to deploy the place recognition server in real-time on a Frontier device with an NVidia Jetson GPU board in future field trials. 
